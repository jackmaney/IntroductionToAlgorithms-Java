\documentclass{article}

\begin{document}

\title{Exercise 2.2-3}
\author{Jack Maney}
\maketitle

Assuming that we're searching a list $l$ for an element $v$, and assume that the probability of any element in $l$ equaling $v$ is $\frac{1}{n}$ (where $l$ has $n$ elements). Let $X$ be the random variable denoting the number of elements of $l$ that needs to be searched to find $v$ via linear search. Then, we have

\[
	E(X) = \sum\limits_{i=1}^n i P(X=i) = 1 \cdot \frac{1}{n} + \sum\limits_{i=2}^n i P(\textrm{the first } i-1 \textrm{ elements of } l \textrm{ are not } v)
\]
\[
	= \frac{1}{n} + \sum\limits_{i=2}^n i \frac{1}{n} \left(1 - \frac{1}{n}\right)^{i-1}  
	= \frac{1}{n} \left(\sum\limits_{i=1}^n i  \left(1 - \frac{1}{n}\right)^{i-1}\right).
\]

Now, consider the polynomial $f(x) = \sum\limits_{i=1}^{n} i (1-x)^{i-1}$. We have,

\[
	\int f(x) dx = C - \sum\limits_{i=1}^n (1-x)^i = C + 1 - \sum\limits_{i=0}^n (1-x)^i 
\]
\[
	= C + 1  - \frac{1 - (1-x)^{n+1}}{1 - (1-x)} = C + 1 - \frac{ 1 - (1-x)^{n + 1}}{x}
\]

\noindent (for $x\not=0$, obviously).

Thus,

\[
	f(x) = \frac{d}{dx}\left( 1 - \frac{ 1 - (1-x)^{n + 1}}{x} \right)
	= \frac{(n+1)x(1-x)^n + (1-x)^{n+1} - 1}{x^2}
\]

So, letting $x = \frac{1}{n}$, we have

\[
	E(X) = \frac{1}{n}\left( \frac{(n+1)(1/n)(1-1/n)^n + (1-1/n)^{n+1}-1}{1/n^2} \right) 
\]
\[
	= n\left(\frac{n+1}{n}\left(1 - \frac{1}{n}\right)^n + \left( 1 - \frac{1}{n}\right)^{n+1} - 1\right) 
	= \left(1 - \frac{1}{n}\right)^n(n+1 + n - 1 - n)
\]
\[
	= \left(1 - \frac{1}{n}\right)^n n
\]

Note that this is asymptotic to $\frac{n}{e}$.

Of course, the best case is when $v$ is the first element of $l$, in which case our running time is $\Theta(1)$. Our worst case scenario is when $v$ is either the last element of $l$ or is not present in $l$ (i.e. we have to search all of $l$ to find $v$), in which case our running time is $\Theta(n)$.
\end{document}