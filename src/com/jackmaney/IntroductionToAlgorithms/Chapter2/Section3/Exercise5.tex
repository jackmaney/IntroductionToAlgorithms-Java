\documentclass{article}

\usepackage{minted}
\usepackage{hyperref}

\begin{document}

\title{Exercise 2.3-5}
\author{Jack Maney}
\maketitle

Here is the main algorithm for Binary Search (found in $BinarySearch.java$), which includes $begin$ and $end$ indices (note that there's an overloaded version with values of 0 and $list.size() - 1$ for $begin$ and $end$, respectively):

\begin{minted}[linenos=true,mathescape=true]{java}

public static <T extends Comparable<T>> int search(AbstractList<T> list
		,T element,int begin,int end){
		
	if(begin > end || begin < 0 || end >= list.size()){
		throw new IllegalArgumentException();
	}
	
	int result = -1;
	
	if(begin == end){
		result = list.get(begin).equals(element) ? begin : -1;
	}
	else{
		int middle = (begin + end) / 2;
		
		int comparison = list.get(middle).compareTo(element);
		
		if(comparison == 0){
			result = middle;
		}
		else if(comparison < 0){ //element is on the right (if it exists)
			result = search(list,element,middle + 1,end);
		}
		else{ //element is on the left (if it exists)
			
			/*
			 * We use upperIndex to avoid the edge case of when
			 * our beginning and ending indices are adjacent.
			 * In such a case, middle == begin, so we can't go
			 * from begin to middle - 1 == begin - 1.
			 */
			int upperIndex = begin + 1 == end ? begin : middle - 1;
			
			result = search(list,element,begin,upperIndex);
		}
	}
	
	return result;
}

\end{minted}


Again, since this algorithm is recursive, we'll compute the cost for one iteration and use induction to view the larger picture.

\bigskip

\begin{tabular}{|c|c|}
\hline 
cost & times \\ 
\hline 
$c_4$ & 1 \\ 
\hline 
$c_8$ & 1 \\ 
\hline 
$c_{10}$ & 1 \\ 
\hline 
$c_{11}$ & 0 or 1 \\ 
\hline 
$c_{14}$ & 1 \\ 
\hline 
$c_{16}$ & 1 \\ 
\hline 
$c_{18}$ & 1 \\ 
\hline 
$c_{19}$ & 0 or 1 \\ 
\hline 
$c_{21}$ & 1 \\ 
\hline 
$T\left(\lfloor \frac{n}{2} \rfloor\right)$ if we reach this branch of code & 1 \\ 
\hline 
$c_{24}$ & 1 \\ 
\hline 
$c_{32}$ & 1 \\ 
\hline 
$T\left(\lfloor \frac{n}{2} \rfloor\right)$ if we reach this branch of code & 1\\ 
\hline 
$c_{38}$ & 1 \\ 
\hline 
 
\end{tabular} 

\bigskip
\bigskip

So, let's consider the worst-case scenario: namely that the element that we seek isn't within our list. We have a total of $\lfloor lg(n) + 1 \rfloor$ iterations of our algorithm (where $n$ is the length of our list) since 

\begin{itemize}
\item $\lfloor lg(n) \rfloor$ is the smallest integer power of 2 that does not exceed $n$--and hence the largest number of sublists that we can pick, and

\item a list with only one element obviously undergoes only one iteration
\end{itemize}

Since the only source of non-constant cost in our binary search algorithm above is due to recursion, it follows that the worst case running time is
\[
	\Theta\left(\lfloor lg(n) + 1 \rfloor\right) = \Theta(lg(n) + 1) = \Theta(lg(n)).
\]

Further, as empirical verification, here are two benchmarks (provided by \href{Perfidix}{http://disy.github.io/perfidix/}) for worst case binary searches (the code is in $BinarySearchBenchmark.java$). 

Here are the benchmark results for 100 repetitions of binary search over a list of 1000 integers:

\begingroup
\fontsize{8pt}{8pt}\selectfont
\begin{verbatim}

|= Benchmark ====================================================================================|
| -                     | unit | sum   | min   | max   | avg   | stddev | conf95        | runs   |
|========================================== TimeMeter ===========================================|
|. BinarySearchBenchmark ........................................................................|
| binarySearchBenchmark | ms   | 17.73 | 00.11 | 01.50 | 00.18 | 00.15  | [00.12-00.23] | 100.00 |
|_ Summary for BinarySearchBenchmark ____________________________________________________________|
|                       | ms   | 17.73 | 00.11 | 01.50 | 00.18 | 00.15  | [00.12-00.23] | 100.00 |
|------------------------------------------------------------------------------------------------|
|=============================== Summary for the whole benchmark ================================|
|                       | ms   | 17.73 | 00.11 | 01.50 | 00.18 | 00.15  | [00.12-00.23] | 100.00 |
|========================================== Exceptions ==========================================|
|================================================================================================|

\end{verbatim}
\endgroup

And here are the benchmarks when increased to a list of 10000 integers (one order of magnitude greater):

\begingroup
\fontsize{8pt}{8pt}\selectfont
\begin{verbatim}

|= Benchmark ====================================================================================|
| -                     | unit | sum   | min   | max   | avg   | stddev | conf95        | runs   |
|========================================== TimeMeter ===========================================|
|. BinarySearchBenchmark ........................................................................|
| binarySearchBenchmark | ms   | 19.72 | 00.11 | 02.15 | 00.20 | 00.21  | [00.11-00.27] | 100.00 |
|_ Summary for BinarySearchBenchmark ____________________________________________________________|
|                       | ms   | 19.72 | 00.11 | 02.15 | 00.20 | 00.21  | [00.11-00.27] | 100.00 |
|------------------------------------------------------------------------------------------------|
|=============================== Summary for the whole benchmark ================================|
|                       | ms   | 19.72 | 00.11 | 02.15 | 00.20 | 00.21  | [00.11-00.27] | 100.00 |
|========================================== Exceptions ==========================================|
|================================================================================================|

\end{verbatim}
\endgroup

\end{document}