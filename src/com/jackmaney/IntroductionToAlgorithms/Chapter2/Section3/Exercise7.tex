\documentclass{article}

\usepackage{minted}
\usepackage{hyperref}

\begin{document}

\title{Exercise 2.3-7}
\author{Jack Maney}
\maketitle

The code integrating binary search within insertion sort can be found in the class $SumSearch$. Again, we consider a worst-case running time--namely, that there are no two integers in the set $S$ whose sum is the target, $x$.

Again, we look at the $search$ method, line by line, and compute the costs, keeping in mind that the Binary Search algorithm has a worst-case running time of $\Theta(lg(n))$ and that Merge Sort has a worst-case running time of $\Theta(n lg(n))$.

\begin{minted}[linenos=true,mathescape=true]{java}

public static Set<Integer> search(Set<Integer> set,int x){
		
		if(set.isEmpty()){
			throw new IllegalArgumentException();
		}
		
		ArrayList<Integer> list = new ArrayList<>();
		list.addAll(set);
		Set<Integer> result = null;
		
		MergeSort.sort(list);
		
		for(int i = 0; i < list.size(); i++){
			
			int y = list.get(i);
			int index = BinarySearch.search(list, new Integer(x - y));
			
			if(index >= 0 && index != i){
				result = new HashSet<Integer>();
				result.add(y);
				result.add(list.get(index));
				break;
			}
			
		}
		
		return result;
	}
\end{minted}

\begin{tabular}{|c|c|}
\hline 
cost & times \\ 
\hline 
$c_3$ & 1 \\ 
\hline 
$c_7$ & 1 \\ 
\hline 
$n$ (line 8)& 1 \\ 
\hline 
$c_9$ & 1 \\ 
\hline 
$n lg(n)$ (mergesort) & 1 \\ 
\hline 
$c_{15}$ & $n$ \\ 
\hline 
$lg(n)$ (binary search) & $n$ \\
\hline
$c_{18}, c_{19}, c_{20}$ and $c_{21}$ & $n$ \\
\hline 
\end{tabular} 

\bigskip

\noindent It follows that the worst-case running time is $\Theta(n lg(n))$.

\end{document}