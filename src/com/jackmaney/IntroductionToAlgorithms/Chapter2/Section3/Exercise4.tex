\documentclass{article}

\usepackage{minted}

\begin{document}

\title{Exercise 2.3-4}
\author{Jack Maney}
\maketitle

For the practice, let's break down the overall running time of the recursive insertion sort, as found in

\[
	com.jackmaney.IntroductionToAlgorithms.sort.RecursiveInsertionSort.java
\]


First, we look at the running time of the $insert$ method. Recall that this method assumes that the elements of $list$ from index 0 to $index - 1$ are sorted, and inserts the element $index$ into the sorted array $list[0..(index - 1)]$.

\begin{minted}[linenos=true,mathescape=true]{java}

public static <T extends Comparable<T>> void insert(AbstractList<T> list, int index){
		
	if(index < 0 || index >= list.size()){
		throw new IllegalArgumentException();
	}
	
	if(index > 0){
		T elementToInsert = list.get(index);
		
		for(int i = index - 1; i >= 0; i--){
			
			if(elementToInsert.compareTo(list.get(i)) <= 0){
				list.set(i+1,list.get(i));
				
				if(i == 0){
					list.set(0,elementToInsert);
				}
			}
			else{
				list.set(i+1,elementToInsert);
				break;
			}
		}
	}
}

\end{minted}

As before, we will denote by $c_i$ a constant running time for line $i$. Let $t$ denote the number of times that the loop in lines 10--23 is run (note that $1\leq t \leq index$)

\begin{tabular}{|c|c|}
\hline 
cost & times \\ 
\hline 
$c_3$ & 1 \\ 
\hline 
$c_7$ & 1 \\ 
\hline 
$c_8$ & 1 \\ 
\hline 
$c_{10}$ & $t+1$ \\ 
\hline 
$c_{12}$ & $t$ \\ 
\hline 
$c_{13}$ & Either $t$ or $t-1$ \\ 
\hline 
$c_{15}$ & Either $t$ or $t-1$ \\ 
\hline 
$c_{16}$ & Either $t$ or $t-1$ \\ 
\hline 
$c_{19}$ & $t$ \\
\hline
$c_{20}$ & 1 \\ 
\hline 
$c_{21}$ & 1 \\ 
\hline 
\end{tabular} 

\bigskip
\bigskip

Therefore, there exist constants $a$, and $b$ such that the total running time is

\[
	T(n,index) = a t + b
\]

\noindent and hence the $insert$ method has a running time anywhere from $\Theta(1)$ to $\Theta(index)$.

We now look at the $sort$ method, which has as arguments not only a list, but an index upon which to use recursion:

\begin{minted}[linenos=true,mathescape=true]{java}

public static <T extends Comparable<T>> void sort(AbstractList<T> list, int index){
	if(index < 0 || index >= list.size()){
		throw new IllegalArgumentException();
	}
	
	if(index > 0){
		sort(list, index - 1);
		insert(list,index);
	}
	
}

\end{minted}

We will compute the total running time--denoted by $T(n,index)$ for running this method once. We will also denote by $t_i$ the running time for the $insert$ method in line 8 ($1 \leq t_i \leq index$).

\begin{tabular}{|c|c|}
\hline 
cost & times \\ 
\hline 
$c_2$ & 1 \\ 
\hline 
$c_6$ & 1 \\ 
\hline 
$T(n,index - 1)$ & 1 \\ 
\hline 
$t_i$ & 1 \\ 
\hline 
\end{tabular} 

So, there exist constants $a$, $b$, and $c$ such that
\[
	T(n,index) = aT(n,index - 1) + b t_i + c.
\]

For the sake of notational convenience, let us denote $index$ by $i$ for the moment. Expanding the recurrence relation above--and using a simple argument by induction--one can show that there exist constants $a_0, a_1,a_2,\cdots,a_{i-1}$ such that
\[
	T(n,i) = a_0 + a_1 t_1 + a_2 t_2 + \cdots + a_i t_i
\]


Finally, we consider the overloaded $sort$ method that recursively sorts the entire list:

\begin{minted}[linenos=true,mathescape=true]{java}

public static <T extends Comparable<T>> void sort(AbstractList<T> list){
	sort(list,list.size() - 1);
}

\end{minted}

And we can write the running time of this method as
\[
	T(n) = a_0 + a_1 t_1 + a_2 t_2 + \cdots + a_n t_n
\]

\noindent where the $a_j$ are constants and $t_j$ are as defined above.

Now, to specifically address the question, the worst case scenario is when the list that we're provided is in reverse sorted order. In that case, $t_j = j$ for all $1 \leq j \leq n$, and we have a running time of $\Theta(n^2)$. 

\end{document}